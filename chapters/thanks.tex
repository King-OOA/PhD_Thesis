
\begin{thanks}

值此论文完成之际,谨向所有关心和帮助过我的老师,同学,朋友,家人表示衷
心的感谢!

首先,衷心感谢我的恩师王宇平教授对我的淳淳教诲和悉心关怀。三年来,王老
师以全面系统的专业知识和高超的专业研究水平对我毫无保留地倾囊相授,并在
本论文的选题、设计、实施和写作等全过程给予具体指导。在学术上,王老师仰
之弥高钻之弥深的学术造诣、泰而不骄矜而不争的为人风范、居之不倦行之以忠
的敬业精神和博学笃志切问近思的治学态度都给学生留下深刻印象。师者,传道
授业解惑也,的言传身教已经并将继续对学生的专业发展起极为重要的作用。在
生活上,王老师也给予我以无微不至的关怀和帮助。在此论文完成之际,谨向王
老师致以最衷心的感谢和诚挚的祝福!

同时,衷心感谢我的师母张爱华老师,在这么多年的学习和生活中,张老师也给
予我很大的关心和帮助!

学习期间,我得到了同门师兄师姐代才,岳伟,杜辉,叶苗,王晓丽,魏静萱,
范磊, 以及薛醒思,张磊,任爱红,魏飞, 还有310实验室的宣贺君,王雨溕,
刘海燕,刘俊华,仝武宁,魏士伟,卫珍,景祯彦,胡丽娟,李省委等博士生和
硕士生的多方面帮助,与他们的探讨也是我受益匪浅!在此对他们表示深深的感
谢!

还要感谢我的好友陆旭和张涛,他们也在学习和生活上给予我很大的帮助。

最后要特别感谢我的爱人王英姿女士,以及我的父母和亲人,他们的理解,支持,
和关心是我最坚强的后盾!感谢他们为我所付出的一切!


彭展,2017年8月,西安电子科技大学

\end{thanks}
