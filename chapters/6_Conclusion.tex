\chapter{总结与展望}
\label{chap:Conclusion}

本章将对全文工作进行总结, 然后提出进一步的研究方向.

\section{全文总结}

本文针对序列(字符串)挖掘领域中的三个重要问题,即多模式(字符串)匹配问
题,后缀排序问题,以及序列的最长公共子序列问题,进行了较为探索和研究,
主要工作包括以下三个方面:

\begin{enumerate}
\item 多模式匹配。 现有的基于内存的多模式匹配算法为模式集所构造的数据结
  构鲁棒性较差,性能易受到模式集自身特性(尤其是最短模式串长)的影响,同
  时可伸缩性较差,在处理大规模模式集时,性能往往无法满足实际需求。针对
  此问题,本文设计并实现了一种高效的多模式匹配引擎。 该引擎包括过滤与核
  实两个模块:过滤模块基于位图结构,所有操作均基于底层位运算,因此能够
  快速地过滤掉文本串中不可能出现匹配的位置;对每一个潜在的匹配位置,调
  用核实模块来确认是否有模式串出现。 核实模块基于一种被称为“自适应匹配
  树”的树形结构,树中的每个节点都保存了模式集的一部分片段,节点内部的
  存储结构将根据自身所保存的模式集片段的特征(即片段长度和片段数量)进行
  自适应地调整,以达到时间效率和空间效率的最佳平衡。 由于每个节点的自适
  应性,使得对于任何特性的模式集,所构造的自适应匹配树都能够保持最高效
  的形态。因此,相比现有算法拥有更好的鲁棒性和可伸缩性。

  另外,对实际中广泛使用的多模式匹配算法---Wu-Manber(WM)算法进行了改进,
  提高了其在地处理较大规模模式集时的效率。不同于WM算法每次都选取模式串
  的前lsp(即最短模式串长)个字符作为其特征串,改进算法通过动态地选取每个
  模式串的特征串,使得模式串能够更加均匀地分布于哈希表中;同时,为哈希
  表的模式串链表设计了索引表,通过在索引表上进行二分查找,能够进一步提
  高算法的搜索速度。实验证实,这两项改进有效地提升了算法在处理较大规模
  模式集时的性能。

\item 后缀排序。基于传统的qsufsort算法框架,提出了一种改进型的后缀排序
  算法---dsufsort。 在传统qsufsort算法的每一轮中,所有后缀都将依据定长
  前缀(即在第$k$轮中,根据每个后缀长为$2^k$的后缀)被排序,这意味着
  前$2^k$个字符都相同的后缀,无法在第$k$轮中被确定顺序,这样,对于那些
  具有很长公共前缀的后缀,需要许多轮才能被确定顺序。为了解决该问
  题,dsufsort算法将记录并维护后缀数组中每个未排序桶的深度,在每一轮排
  序中,将根据待排序桶的深度对其中的后缀进行排序,
  这使得在第$k$轮中,后缀可以基于长度超过$2^k$的前缀被排序,从而那些
  前$2^k$个字符相同的后缀便可在第$k$轮中被确定顺序, 因此,dsufsort算法
  仅需要较少的轮数就可以完成排序。 此外,由于桶的深度具有累加性,因此,
  对于具有很长公共前缀的后缀,dsufsort算法可以更快速地完成排序。

\item 求序列的最长公共子序列。 现有算法在求解最长公共子序列问题时通常需
  要构建有向无环图,在图构造好之后通过搜索其中的最长路径来构建相应的最
  长公共子序列。然而,由于图中的节点数量过多,会导致大量的内存消耗,同
  时,在大规模图中搜索最长路径会花费较长的运行时间。 针对此问题,本文提
  出了一种新的层次化图模型---Leveled-DAG,及其相应的构建算法。 不同于现
  有的算法在构造有向无环图时,需要保存所有产生的节点,并在图构造好之后
  通过搜索其中的最长路径来构建相应的最长公共子序列, Leveled-DAG模型可
  以在建图的过程中实时地构建目标序列的最长公共子序列,并及时删除那些对
  构建最长公共子序列没有任何影响的无用节点。在任一时刻,Leveled-DAG只需
  保存最新产生的一层节点以及前面产生的入度不为0的节点,并且,随着构建过
  程的进行,图中的节点数将会逐渐减少,最终将仅剩余一个节点,所有目标序
  列的最长公共子序列都保存在该节点中。 得益于实时地构造最长公共子序列及
  删除无用节点,Leveled-DAG相比现有算法在时间和空间效率上都有较大提升。
  \end{enumerate}

\section{工作展望}

本文对序列挖掘中的三个关键问题:多模式匹配、后缀排序、以及最长公共子序
列问题,进行了较为深入的研究,分别提出了几种新的方法。然而,这些方法仍
有不成熟之处,在许多方面都值得进一步研究:

\begin{enumerate}
\item 对于自适应匹配树,研究更加高效的树节点的结构,可以从整体上提高核
  实过程的效率;同时,对自适应策略所依赖的各种参数,需要进行更加精细的
  设置。
\item 对于dsufsort算法,对桶的处理顺序会对算法的性能造成一定影响, 所以
  需要进一步研究最优的桶处理顺序。 其次, 用来记录桶深的数组可能会非常稀
  疏, 这会增加空间开销, 所以研究稀疏数组压缩技术也非常重要。
\item 对于Leveled-DAG 模型,如何更加高效地删除过时节点将极大的提高算法
  的性能;同时,由于算法需要大量的内存分配, 调整大小, 以及释放操作,因
  此需要寻找更高效的内存管理函数来取代由标准库提供的内存分配函数。

\end{enumerate}
